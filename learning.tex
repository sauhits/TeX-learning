\documentclass[a4]{jsarticle}
\usepackage{amsmath}
\begin{document}
\title{練習1}

複素関数$f(z)=f(x+iy)=u(x,y)+iv(x,y)$が点$z_0=x_0+iy_0$におい
て正則であるための必要十分条件は,$z_0$のある$ε$近傍$Δ(z_0,ε)$において以下のコーシー・リーマン方程式を満たすことである。

\[\frac{\delta u}{\delta x}=\frac{\delta v}{\delta y}\]
\[\frac{\delta u}{\delta y}=-\frac{\delta v}{\delta x}\]


\title{練習2}
$c(t)=(x(t),y(t),z(t))$によって与えられる空間曲線$c$の$c(0)$を始点として$c(t)$までの弧長を$s(t)$とすると

\[s(t)=\int_{0}^{t} \sqrt{\Big(\frac{dx}{dt}\Big)^2 + (\frac{dy}{dt})^2 + (\frac{dz}{dt})^2} dt\]

と表される。


\title{練習3}
$関数fが開区間I上でn回微分可能であるとする.このとき,a,b \in Iに対し,$

\[f(b)=f(a)+\frac{f'(a)}{1!}(b-a)+\frac{f''(a)}{2!}(b-a)^2+\cdots +\frac{f^{(n-1)}(a)}{(n-1)!}(b-a)^{n-1} +R_n (c)\]
$を満たすcがaとbの間に存在する.$


\title{練習4}
$m次正方行列$

\[J(\alpha ,m)= \left[\begin{array}{rrrrr}
  \alpha & 1 & 0 & \cdots & 0 \\
  0 & \alpha & 1 & \ddots & \vdots \\
  \vdots & \ddots & \ddots & \ddots & 0 \\
  \vdots &  & \ddots & \ddots & 1 \\
  0 & \cdots & \cdots & 0 & \alpha
\end{array}\right]
\]

$をJordan細胞と呼ぶ.正方行列Aが正則行列Pによって$

\begin{align}
  P^{-1}AP &= J(\alpha_1,m_1)\oplus J(\alpha_2,m_2)\oplus\cdots\oplus J(\alpha_k,m_k)\\
  &=\left[\begin{array}{rrrr}
    J(\alpha_1,m_1) & & & \\
    & J(\alpha_2,m_2) & & \\
    & & \ddots & \\
    & & & J(\alpha_k,m_k)
  \end{array}\right]
\end{align}

$とjordan細胞の直和になるとき,これをAのJordan標準形と呼ぶ.$
\end{document}