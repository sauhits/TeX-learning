\documentclass[a4]{jsarticle}
\begin{document}
\title{練習1}

複素関数$f(z)=f(x+iy)=u(x,y)+iv(x,y)$が点$z_0=x_0+iy_0$におい
て正則であるための必要十分条件は,$z_0$のある$ε$近傍$Δ(z_0,ε)$において以下のコーシー・リーマン方程式を満たすことである。

\[\frac{\delta u}{\delta x}=\frac{\delta v}{\delta y}\]
\[\frac{\delta u}{\delta y}=-\frac{\delta v}{\delta x}\]


\title{練習2}
$c(t)=(x(t),y(t),z(t))$によって与えられる空間曲線$c$の$c(0)$を始点として$c(t)$までの弧長を$s(t)$とすると

\[s(t)=\int_{0}^{t} \sqrt{\Big(\frac{dx}{dt}\Big)^2 + (\frac{dy}{dt})^2 + (\frac{dz}{dt})^2} dt\]

と表される。

\end{document}